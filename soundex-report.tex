% -*-latex-*-
% Typical article template

\documentclass[a4paper,11pt]{article}
\usepackage{amsmath}
\usepackage{amstext}
\usepackage{amsbsy}
\title{ NLP Homework 1 }

\author{Eray \"Ozkural}

\date{\today}

\begin{document}

\maketitle

I have implemented the Soundex algorithm using ocaml language. First,
I have modified an old imperative directed graph implementation of
mine so that it supports labels of any type and multi-graph instead of
a simple graph.  \footnote{Since the FST diagram has multiple edges
  among a pair of vertices}

The transition function is represented by this graph, and it is
exactly the same thing as the diagram of an FST. The only remaining
variables are the starting state and the set of final states which are
encapsulated in a record. The FST and other code has been implemented
as separate modules which can be re-used easily. The transducer
algorithm has been programmed as a simple recursive function which
finds a matching prefix among the labels of adjacent edges of the
given state in the graph of transition function, emits the output
symbol, and calls itself recursively to construct a list of output
symbols which are subsequently concatenated to obtain the output
string.

The design of the soundex algorithm is achieved by cascading
transducers designed for each step. The design of step a and b is
straightforward but the other steps require a bit of
explanation. In all of the transducers, I was able to express the FSTs
at a very high level since ocaml has higher order functions such as
map and iter for List and Array types. For
instance in fstb, I have a line that is:
\begin{verbatim}
   List.iter (fun x -> add d (1,1) (x,"1")) ["b";"f";"p";"v"];
\end{verbatim}
These constructs reduce the code size considerably.

The design of step c is achieved as follows, for each digit we have a
separate state $10+i$ that keeps track of repetitions and does not
emit anything as long as the same digit is read. Whenever the input
comes to another digit or to end of input, the state's character
is emitted.

In the design of d, we apply usual finite state methods. There is a
state for each of the digits output. FST terminates after the state 4
corresponding to 3rd digit output. If end of string is seen before,
padding zeroes are added appropriately.

To simplify implementation, character ``\#'' is treated as sentinel
for the input.

\end{document}
